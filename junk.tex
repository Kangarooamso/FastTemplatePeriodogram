


%We compute useful quantities to simplify later calculations:

%\begin{align}
%\partial_b M(t) = \sum_n^H &n\left(c_nU_{n-1}(b) + s_n\frac{T_n(b)}{\sqrt{1 - b^2}}\right)\comt[n]\\
%		&+n\left(s_nU_{n-1}(b) - c_n\frac{T_n(b)}{\sqrt{1 - b^2}}\right)\somt[n]
%\end{align},

%which we can rewrite using



We can then rewrite

\begin{equation}
\partial_b M(t) = \sum_n^H A'_n(b)\comt + B'_n(b)\somt.
\end{equation}



\begin{align}
\label{eq:msquared}
M^2(t) = \sum_{n=1}^H\sum_{m=1}^H\big[A_n(b)A_m(b)\comt[n]\comt[m] + 2A_n(b)B_m(b)\comt[n]\somt[m] + B_n(b)B_m(b)\somt[n]\somt[m]\big]
\end{align}

\begin{align}
\label{eq:mdbm}
M(t)\partial_b M(t) = \sum_{n=1}^H\sum_{m=1}^H\big[&A_n(b)A'_m(b)\comt[n]\comt[m] + A_n(b)B'_m(b)\comt[n]\somt[m] \\
			          &+ B_n(b)A'_m(b)\somt[n]\comt[m] + B_n(b) B'_n(b)\somt[n]\somt[m]\big]
\end{align}

The next set of values are pre-computed sums that will be solved efficiently with (non-equispaced) fast Fourier transforms:

\begin{eqnarray}
\Cc   &\equiv& \sum_i w_i \cos n\omega t_i\\
\Ss   &\equiv& \sum_i w_i \sin n\omega t_i\\
\CCc &\equiv& \sum_i w_i \cos n\omega t_i\cos m \omega t_i\\
\SSs &\equiv& \sum_i w_i \sin n\omega t_i\sin m \omega t_i\\
\CS &\equiv& \sum_i w_i \cos n\omega t_i\sin m \omega t_i\\
\YC  &\equiv& \sum_i w_i y_i \cos n\omega t_i\\
\YS  &\equiv& \sum_i w_i y_i \sin n\omega t_i
\end{eqnarray}

Now we simplify the system of equations:

\subsection{first eqn}
\begin{align}
0                                    &= \partial_{a} \chi^2 = 2W\sum_i w_i \left(y_i - y(t_i)\right)M(t_i)\\
\sum_i w_i y_i M(t_i)                &= \sum_i w_i M(t_i) \left(aM(t_i) + c\right)\\
\sum_iw_i(y_i - c)M(t_i) &= a\sum_i w_iM^2(t_i)%\\
%\sum_n \left[A_n(b)(\YC - c\Cc) + B_n(b)(\YS - c\Ss)\right] &= a\sum_n\sum_m\big[A_n(b)A_m(b)\CCc + 2A_n(b)B_m(b)\CS \\
%			&+ B_n(b)B_m(b)\SSs\big]
\end{align}

\subsection{second eqn}

\begin{eqnarray}
0 &=& \partial_{b} \chi^2 = 2W\sum_i w_i \left(y_i - y(t_i)\right)\dM(t_i)\\
\sum_i w_i y_i \dM(t_i) &=& a\sum_i w_iM(t_i)\dM(t_i) + c\sum_i w_i\dM(t_i)\\
\sum_i w_i (y_i - c)\dM(t_i) &=& a\sum_iw_i M(t_i)\dM(t_i)
\end{eqnarray}


\subsection{third eqn}

\begin{eqnarray}
0 &=& \partial_{c} \chi^2 = 2W\sum_i w_i \left(y_i - y(t_i | \vec{\theta})\right)\\
\underbrace{\sum_i w_i y_i}_{\equiv\bar{y}} &=& a\sum_i w_i M(t_i) + c\underbrace{\sum_i w_i}_{=1}\\
c &=& \bar{y} - a\sum_i w_i M(t_i)
\end{eqnarray}

\subsection{reducing first and third equations}

\begin{eqnarray}
\sum_iw_i(y_i - c)M(t_i) &=& a\sum_i w_iM^2(t_i)\\
\sum_iw_i\left(y_i - \bar{y} + a\sum_jw_jM(t_j)\right)M(t_i) &=& a\sum_i w_i M^2(t_i)\\
\sum_iw_iy_iM(t_i) - \bar{y}\sum_iw_iM(t_i) + a\left(\sum_iw_iM(t_i)\right)^2 &=& a\sum_i w_i M^2(t_i)\\
\sum_n\left(A_n\underbrace{(\YC - \bar{y}\Cc)}_{\hatYC} + B_n\underbrace{(\YS - \bar{y}\Ss)}_{\hatYS}\right) + a\left(\sum_iw_iM(t_i)\right)^2&=& a\sum_i w_i M^2(t_i)\\
\sum_n\left(A_n\hatYC + B_n\hatYS\right) + a\sum_n\sum_m\left(A_nA_m\Cc[n]\Cc[m] + 2A_nB_m\Cc[n]\Ss[m] + B_nB_m\Ss[n]\Ss[m]\right) &=& a\sum_i w_i M^2(t_i)\\
a\sum_n\sum_m\left(A_nA_m\underbrace{(\CCc - \Cc[n]\Cc[m])}_{\hatCC} 
       + 2A_nB_m\underbrace{(\CS - \Cc[n]\Ss[m])}_{\hatCS} 
       + B_nB_m\underbrace{(\SSs - \Ss[n]\Ss[m])}_{\hatSS}\right)&=& \sum_n\left(A_n\hatYC + B_n\hatYS\right)
\end{eqnarray}



We can now use our result from the third equation and make the following simplification to the first equation:


%\begin{multline}
%\sum_n\left[A_n\underbrace{(\YC - \bar{y}\Cc)}_{\hatYC} + 
%B_n\underbrace{(\YS - \bar{y}\Ss)}_{\hatYS}\right] 
%= a\sum_n\sum_m\left[A_nA_m\underbrace{(\CCc - \Cc\Cc[m])}_{\hatCC} \right.
%\\\left. + 2A_nB_m)\underbrace{(\CS - \Cc\Ss[m])}_{\hatCS}
%   + B_nB_m\underbrace{(\SSs - \Ss\Ss[m])}_{\hatSS} \right]
%\end{multline}

after doing this, we can now solve for $a$ in terms of $b$:

\begin{equation}
a = \frac{\sum_n\left[A_n\hatYC + B_n\hatYS\right]}
{\sum_n\sum_m\left[A_nA_m\hatCC + 2A_nB_m\hatCS + B_nB_m\hatSS\right]}
\end{equation}

\subsection{final expression for $b$}
we also can express $a$ (via the result from the second equation) as

\begin{equation}
a = \frac{\sum_n\left[A_n'\hatYC + B_n'\hatYS\right]}
{\sum_n\sum_m\left[A_n'A_m\hatCC + (A_nB_m' + A_n'B_m)\hatCS + B_n'B_m\hatSS\right]}
\end{equation}

equating this to the previous expression, cross-multiplying, and then simplifying, gives the following expression
\begin{equation}
\begin{split}
\sum_n\sum_m\sum_k\Big[ & A_kB_nA_m'\left(\hatYS[n]\hatCC[mk] - \hatYC[m]\hatCS[kn]\right)
+ A_kA_nB_m'\left(\hatYC[k]\hatCS[nm] - \hatYS[m]\hatCC[nk]\right)
\\ & + A_kB_nB_m'\left(\hatYC[k]\hatSS[nm] - \hatYS[m]\hatCS[kn]\right) 
+ B_kB_nA_m'\left(\hatYS[k]\hatCS[mn] + \hatYC[m]\hatSS[kn]\right)\Big] = 0
\end{split}
\end{equation}

\begin{eqnarray}
A_n(b)A_m(b) &=& \sum_{i=1}^H\sum_{j=1}^i \left(q_1q_2(T_n^j)(T_m^{i-j}) - (q_1p_2(U_{s_nU_{n-1}(b)\sqrt{1 - b^2}\right)b^i
\end{eqnarray}

There are four terms of this triple sum. Each of these terms involves products of Chebyshev polynomials 
of both the first and second kinds and their derivatives (which can be expressed in terms of Chebyshev 
polynomials themselves). Chebyshev polynomials obey product rules, namely

\begin{eqnarray}
2T_n(x)T_m(x) &=& T_{m+n}(x) + T_{|m-n|}(x)\\
2T_n(x)U_m(x) &=& \left\{ 
\begin{array}{ll}
U_{n+m}(x) + U_{m-n}(x) & m \geq n - 1 \\
U_{n+m}(x) - U_{n - m - 2}(x) & m \leq n - 2
\end{array}
\right.
\end{eqnarray}

Which means that we are garuanteed to end up with a \emph{polynomial} expression in $b$ and $\sqrt{1 - b^2}$ 
that is of order $H + H + (H-1) = 3H - 1$. What we want, however, is a polynomial in $b$ only; we can get 
this by rearranging things like so:

\begin{eqnarray}
\sum_{i=0}^{3H-1}a_ib^i + \sum_{i=0}^{3H-1}\sum_{j=0}^{3H-1}A_{ij}b^i\left(1-b^2\right)^{j+1/2} &=& 0\\
\left[(1-b^2)^{1/2}\sum_{i=0}^{3H-1}\sum_{j=0}^{3H-1}A_{ij}b^i\left(1-b^2\right)^{j}\right]^2 &=& \left[-\sum_{i=0}^{3H-1}a_ib^i\right]^2\\
\sum_{i=0}^{2 + 3*(3H-1)} \widetilde{a}_{i} b^i &=& 0
\end{eqnarray}

The resulting polynomial is of order $9H - 1$ or less. This is the first sign that either (1) I've made a mistake somewhere, or, more
optimistically, (2) a LOT of terms cancel, because as we know the $H=1$ case is \emph{linear} in $b$.

As we've already begun to see, after this point things start to get \emph{really} hairy. I'm not very proficient with Mathematica, 
but I think it's probably a necessary tool to make the rest of our work tractable and error-free.

I'm going to provide a couple of additional attempts that I've made to ``simplify'' things, though the algebra is (very) tedious
and (very) unfinished.  

Using the following useful property of Chebyshev polynomials:

\begin{eqnarray}
T_{n+1}(x) &=& xT_n(x) - \sqrt{(1-x^2)\left[1-T_n^2(x)\right]}\\
\sqrt{1-T_n^2(x)} &=& (1-x^2)^{-1/2}\left(xT_n(x) - T_{n+1}(x)\right)
\end{eqnarray}

which assumes $|x|>0$ (let's ignore that for now), we can rewrite the expressions for $A_n$ and $B_n$ as 

\begin{eqnarray}
A_n &=& \left[c_n - s_nb(1-b^2)^{-1/2}\right]T_n(b) + s_n(1-b^2)^{-1/2}T_{n+1}(b) \\
B_n &=& \left[s_n + c_nb(1-b^2)^{-1/2}\right]T_n(b) - c_n(1-b^2)^{-1/2}T_{n+1}(b).
\end{eqnarray} 

To further simplify things, I'll introduce some shorthand notation:

\begin{eqnarray}
Q_n^i = (1-b^2)^{-1/2}\left[\left(\alpha_n^i\sqrt{1-b^2} + \beta_n^ib\right)T_n(b) - \beta_n^iT_{n+1}(b)\right]
\end{eqnarray}

Here, $i\in{A, B}$ and $(\alpha^A,\beta^A) = (c_n, -s_n)$, $(\alpha^B, \beta^B) = (s_n, c_n)$. Now, each of the four terms
in the final sum can be reduced to a \emph{single} expression: $Q_n^iQ_m^j\partial_bQ_m^k$. 

Below is the derivation of $\partial_bQ_n^i$ ({\bf Note: this is the ``long'' version of the derivation. Skip this section for the simpler derivation}):

\newcommand{\ddb}{\frac{\partial}{\partial b}}
\newcommand{\ani}{\alpha_n^i}
\newcommand{\bni}{\beta_n^i}
\begin{align*}
\frac{\partial Q_n^i}{\partial b} &= Q_n^i(1-b^2)^{1/2}\ddb\left((1 - b^2)^{-1/2}\right) \\
 								  &\qquad\quad + (1-b^2)^{-1/2}\Bigg[\ddb\left(\ani(1-b^2)^{1/2} + \bni b\right)T_n(b)\\
 								  &\qquad\quad + \left(\ani(1-b^2)^{1/2} + \bni b\right)\ddb T_n(b) - \beta_n^i\ddb T_{n+1}(b)\Bigg]\\
 								  &= b(1-b^2)^{-1}Q_n^i + (1-b^2)^{-1/2}\Bigg[\left(-\ani b(1-b^2)^{-1/2} + \bni\right)T_n(b) \\
 								  &\qquad\quad + \left(\ani (1-b^2)^{1/2} + \bni b\right)nU_{n-1}(b) - \bni(n+1)U_n(b)\Bigg]\\
 								  &= b(1-b^2)^{-3/2}\left[\left(\ani(1-b^2)^{1/2} + \bni b\right)T_n(b) - \bni T_{n+1}(b)\right]\\
 								  &\qquad\quad + (1-b^2)^{-1/2}\left[\left(\bni - \ani b(1-b^2)^{-1/2}\right)T_n(b) + \left(\ani (1-b^2)^{1/2} + \bni b\right)nU_{n-1}(b) - \bni (n+1)U_n(b)\right] \\
 								  &= \ani \left[\underbrace{b(1-b^2)^{-1}T_n(b) - b(1-b^2)^{-1}T_n(b)}_{0} + nU_{n-1}(b)\right] \\
 								  &\qquad\quad + \bni \Big[b^2(1-b^2)^{-3/2}T_n(b) - b(1-b^2)^{-3/2}T_{n+1}(b) + (1-b^2)^{-1/2}T_n(b) \\
 								  &\qquad\quad\qquad\quad + b(1-b^2)^{-1/2}nU_{n-1}(b) - (n+1)(1-b^2)^{-1/2}U_n(b)\Big]\\
 								  &=\ani nU_{n-1}(b) + \bni(1-b^2)^{-3/2}\Bigg[T_n(b) - bT_{n+1}(b) + (1-b^2)\Big(n\underbrace{\left[bU_{n-1}(b) - U_n(b)\right]}_{-T_n(b)} - U_n(b)\Big)\Bigg]\\
 								  &=\ani nU_{n-1}(b) + \bni(1-b^2)^{-3/2}\Bigg[T_n(b) - bT_{n+1}(b) - (1-b^2)(nT_n(b) - U_n(b)\Bigg]
\end{align*}

We can now apply the identity: $T_{n+1}(b) = bT_n(b) - (1-b^2)U_{n-1}(b)$:

\begin{align*}
\frac{\partial Q_n^i}{\partial b} &=\ani nU_{n-1}(b) + \bni(1-b^2)^{-3/2}\Bigg[T_n(b) - b^2T_n(b) + b(1-b^2)U_{n-1}(b) - (1-b^2)\left(nT_n(b) + U_n(b)\right)\Bigg]\\
								  &=\ani nU_{n-1}(b) + \bni(1-b^2)^{-1/2}\Bigg[\underbrace{T_n(b) + \underbrace{bU_{n-1}(b) - U_n(b)}_{-T_n(b)}}_{0} - nT_n(b)\Bigg]\\
								  &=n(1-b^2)^{-1/2}\left(\ani\sqrt{1-b^2}U_{n-1}(b) - \bni T_n(b) \right)
\end{align*}
\subsubsection{Here's the easier derivation for $\partial_b Q_n^i$}

After writing all of this out, I now realize that it's actually the ``hard'' way; I'll give the shorter version below as a double check.
We can alternatively rewrite $Q_n^i = \ani T_n(b) + \bni\sqrt{1-b^2}U_{n-1}(b)$.

\begin{align*}
\frac{\partial Q_n^i}{\partial b} &= \ani \ddb T_n(b) + \bni\left[\sqrt{1-b^2}\ddb U_{n-1}(b) - b(1-b^2)^{-1/2}U_{n-1}(b)\right]\\
								  &= \ani n U_{n-1}(b) + \bni\sqrt{1-b^2}\left[\frac{bU_{n-1}(b) - nT_n(b) - bU_{n-1}(b)}{1-b^2}\right]\\
								  &= n\left[\ani U_{n-1}(b) - \bni(1-b^2)^{-1/2}T_n(b)\right]
\end{align*}


The next steps are the following: 

\begin{enumerate}
\item Derive and simplify the expression for $Q_n^iQ_m^j\partial_bQ_l^k$ as a polynomial in $b$ and $\sqrt{1-b^2}$. 
\item Use this result to generate a polynomial in $b$ and $\sqrt{1-b^2}$ from the final equation
\item From this, compute the coefficients of $\sum_i \widetilde{a}_i b^i = 0$
\item Profit.
\end{enumerate}


\newcommand{\amj}{\alpha_m^j}
\newcommand{\alk}{\alpha_l^k}
\newcommand{\bmj}{\beta_m^j}
\newcommand{\blk}{\beta_l^k}

We start by computing $Q_n^iQ_m^j$
\begin{align*}
Q_n^iQ_m^j &= \left(\ani T_n(b) + \bni(1-b^2)^{1/2}U_{n-1}(b)\right)\left(\amj T_m(b) + \bmj(1-b^2)^{1/2} U_{m-1}(b)\right) \\
	   &= \ani\amj T_n(b)T_m(b) + \ani\bmj(1-b^2)^{1/2}T_n(b)U_{m-1}(b) \\
	   &\qquad\quad \bni\amj(1-b^2)^{1/2}U_{n-1}(b)T_m(b) + \bni\bmj(1-b^2)U_{n-1}(b)U_{m-1}(b)
\end{align*}

Multiplying by $\partial_b Q_l^k$ yields:

\begin{align*}
Q_n^iQ_m^j\ddb Q_l^k &= l\Bigg[\ani\amj\alk T_n(b)T_m(b) U_{l-1}(b) - \ani\amj\blk(1-b^2)^{-1/2}T_n(b) U_{m-1}(b) \\
		     &\qquad\quad \ani\bmj\alk(1-b^2)^{1/2}T_n(b) U_{m-1}(b) U_{l-1}(b) - \ani\bmj\blk T_n(b) U_{m-1}(b) T_l(b) \\
		     &\qquad\quad \bni\amj\alk(1-b^2)^{1/2}U_{n-1}(b) T_m(b) U_{l-1}(b) - \bni\amj\blk U_{n-1}(b) T_m(b) T_l(b) \\
		     &\qquad\quad \bni\bmj\alk(1-b^2)U_{n-1}(b) U_{m-1}(b) U_{l-1}(b) - \bni\bmj\blk (1-b^2)^{1/2}U_{n-1}(b)U_{m-1}(b)T_l(b)\Bigg]
\end{align*}

At this point, in order to obtain the coefficients for the polynomial, we need to expand each product of Chebyshev polynomials and collect like terms.

But I will stop here for now to see if you have any ideas for possible shortcuts, or at least a more efficient/reliable way of deriving the coefficients.

\end{document}
